\documentclass{article}
\input{fitch.sty}
\input{bussproofs.sty}
\usepackage{fullpage}
\usepackage{graphicx}

\title{Tarea 2}
\date{02/02/2016}
\author{Alejandra Cordero (12-10645) y Pablo Maldonado (12-10561)}

\begin{document}

\pagenumbering{gobble}
\begin{titlepage}

\begin{center}
\includegraphics[width=3cm, height=2cm]{logo.png}\\[0.4cm]
\large{Universidad Sim\'on Bol\'ivar}\\
\large{Departamento de Computaci\'on y Tecnolog\'ia de la Informaci\'on}\\
\large{Laboratorio de Lenguajes de Programaci\'on - CI3661}\\[7cm]
\Huge\textbf{Tarea 2}\\[7cm]
\begin{raggedleft}
\large{Alejandra Cordero - 12-10645}\\
\large{Pablo Maldonado -12-10561}\\
\large Abril - Julio 2016
\end{raggedleft} 

\end{center}

\end{titlepage}

\pagenumbering{arabic}

\paragraph{}

1. Demuestre los teoremas TD y DM por Fitch (teorema de la disyunci\'on y teorema de De Morgan respectivamente)


\begin{itemize}

\item $p \rightarrow q	
\leftrightarrow \neg p \lor q$

\[
\begin{nd}
\hypo {1} {P\vee Q}
\open
\hypo {2} {P}
\have [\vdots] {3} {\vdots}
\have [n][-1] {4} {A\wedge B}
\close
\open
\hypo {5} {Q}
\have [\vdots] {6} {\vdots}
\have [m] {7} {A\wedge B}
\close
\have {8} {A\wedge B} \oe{1,2-(4),5-7}
\have [\vdots] {9} {\vdots}
\have [][100] {10} {A} \ae{8}
\end{nd}
\]

\item $\neg (p \land q) \leftrightarrow \neg p \lor \neg q$
	
\[
\begin{nd}
\hypo {1} {P\vee Q}
\open
\hypo {2} {P}
\have [\vdots] {3} {\vdots}
\have [n][-1] {4} {A\wedge B}
\close
\open
\hypo {5} {Q}
\have [\vdots] {6} {\vdots}
\have [m] {7} {A\wedge B}
\close
\have {8} {A\wedge B} \oe{1,2-(4),5-7}
\have [\vdots] {9} {\vdots}
\have [][100] {10} {A} \ae{8}
\end{nd}
\]


\end{itemize}

\paragraph{}

2. Las reglas en PROLOG son f\'ormulas en FNC que adem\'as deben ser cl\'ausulas de Horn ?`Qu\'e significa este concepto? Adem\'as deben estar "skolemizadas"?`Por qu\'e? ?`Qu\'e hace ese procedimiento? Por \'ultimo existe la unificaci\'on ?`En qu\'e consiste?

\paragraph{Cl\'ausula de Horn}\mbox{}\\

En primer lugar, se desea saber qu\'e es una Clausula de Horn. Para ello, hace falta definir primero los siguientes t\'erminos:

\begin{itemize}

\item Literal: Es una f\'ormula at\'omica o su negaci\'on. Por ejemplo, son literales: $p, \neg q, z, \neg t$, etc.
\item Cl\'ausula: Disyunci\'on de literales. Ejemplo: $ p \lor \neg q \lor \neg t \lor ... \lor z $ 

\end{itemize}

Luego, una Clausula de Horn no es m\'as que una cl\'ausula, en la cual existe exactamente un literal positivo. Por ejemplo: $ \neg p \lor \neg q \lor ... \lor \neg t \lor u$ 

\paragraph{Skolemizaci\'on}\mbox{}\\





\paragraph{Unificaci\'on}\mbox{}\\

Procedimiento mediante el cual Prolog "matches" dos t\'erminos.


\paragraph{}

3. Demuestre por Fitch y Resoluci\'on por refutaci\'on

\begin{itemize}

	\item $\{A \rightarrow (( B \lor \neg A) \land D), E \rightarrow (\neg B \land \neg C), (B \land G) \rightarrow (\neg C \land \neg D)\} \vdash G  \rightarrow ((A \lor C) \rightarrow ( \neg B \land \neg E))$

	\begin{itemize}
	
	\item M\'etodo de Fitch:
	
	
	\[
	\begin{nd}
	\hypo {1} {P\vee Q}
	\open
	\hypo {2} {P}
	\have [\vdots] {3} {\vdots}
	\have [n][-1] {4} {A\wedge B}
	\close
	\open
	\hypo {5} {Q}
	\have [\vdots] {6} {\vdots}
	\have [m] {7} {A\wedge B}
	\close
	\have {8} {A\wedge B} \oe{1,2-(4),5-7}
	\have [\vdots] {9} {\vdots}
	\have [][100] {10} {A} \ae{8}
	\end{nd}
	\]
	
	
	
	
	
	
	
	
	
	
	
	
	\item M\'etodo de Resoluci\'on por Refutaci\'on:
	
		\paragraph{}
		En primer lugar, es necesario transformar la expresi\'on a la FNC correspondiente 	
	
	
	
		\begin{prooftree}
		\AxiomC{$p \lor q $}
		\AxiomC{$p \lor \neg q $}
		\AxiomC{C}
		\BinaryInfC{D}
		\BinaryInfC{E}
		\end{prooftree}
	
	\end{itemize}		
	
	\item  $\neg p \rightarrow (q \lor r)) \vdash ((\neg p \rightarrow \neg q) \lor r)$ 
	
	\begin{itemize}
	
	\item M\'etodo de Fitch:
	
	\item M\'etodo de Resoluci\'on por Refutaci\'on:
		\paragraph{}
		En primer lugar, es necesario transformar la expresi\'on a la FNC correspondiente
	
		\begin{prooftree}
		\AxiomC{$p \lor q $}
		\AxiomC{$p \lor \neg q $}
		\AxiomC{C}
		\BinaryInfC{D}
		\BinaryInfC{E}
		\end{prooftree}
	
	\end{itemize}	
	
	
	
\end{itemize}


\paragraph{Referencias} 

\begin{itemize}

\item 

\end{itemize}

\end{document}


