\documentclass{article}
\input{fitch.sty}
\input{bussproofs.sty}

\begin{document}

\paragraph{}

1. Demuestre los teoremas TD y DM por Fitch (teorema de la disyunci\'on y teorema de De Morgan respectivamente)

\paragraph{}

2. Las reglas en PROLOG son f\'ormulas en FNC que adem\'as deben ser cl\'ausulas de Horn ?`Qu\'e significa este concepto? Adem\'as deben estar "skolemizadas"?`Por qu\'e? ?`Qu\'e hace ese procedimiento? Por \'ultimo existe la unificaci\'on ?`En qu\'e consiste?

\begin{itemize}

\item $p \rightarrow q	
\leftrightarrow \neg p \lor q$

\[
\begin{nd}
\hypo {1} {P\vee Q}
\open
\hypo {2} {P}
\have [\vdots] {3} {\vdots}
\have [n][-1] {4} {A\wedge B}
\close
\open
\hypo {5} {Q}
\have [\vdots] {6} {\vdots}
\have [m] {7} {A\wedge B}
\close
\have {8} {A\wedge B} \oe{1,2-(4),5-7}
\have [\vdots] {9} {\vdots}
\have [][100] {10} {A} \ae{8}
\end{nd}
\]

\item $\neg (p \land q) \leftrightarrow \neg p \lor \neg q$
	
\[
\begin{nd}
\hypo {1} {P\vee Q}
\open
\hypo {2} {P}
\have [\vdots] {3} {\vdots}
\have [n][-1] {4} {A\wedge B}
\close
\open
\hypo {5} {Q}
\have [\vdots] {6} {\vdots}
\have [m] {7} {A\wedge B}
\close
\have {8} {A\wedge B} \oe{1,2-(4),5-7}
\have [\vdots] {9} {\vdots}
\have [][100] {10} {A} \ae{8}
\end{nd}
\]




\end{itemize}

\paragraph{}

3. Demuestre por Fitch y Resoluci\'on por refutaci\'on

\begin{itemize}

	\item $\{A \rightarrow (( B \lor \neg A) \land D), E \rightarrow (\neg B \land \neg C), (B \land G) \rightarrow (\neg C \land \neg D)\} \\ \vdash G  \rightarrow ((A \lor C) \rightarrow ( \neg B \land \neg E))$
	
	\item  $\neg p \rightarrow (q \lor r)) \vdash ((\neg p \rightarrow \neg q) \lor r)$ 
	
\end{itemize}

\begin{prooftree}
\AxiomC{$p \lor q $}
				\AxiomC{$p \lor \neg q $}
							\AxiomC{C}
				\BinaryInfC{D}
		\BinaryInfC{E}
\end{prooftree}


\end{document}


