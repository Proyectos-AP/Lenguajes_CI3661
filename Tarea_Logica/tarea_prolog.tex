\documentclass{article}
\input{fitch.sty}
\input{bussproofs.sty}
\usepackage{fullpage}
\usepackage{graphicx}

\title{Tarea 2}
\date{02/02/2016}
\author{Alejandra Cordero (12-10645) y Pablo Maldonado (12-10561)}

\begin{document}

\pagenumbering{gobble}
\begin{titlepage}

\begin{center}
\includegraphics[width=3cm, height=2cm]{logo.png}\\[0.4cm]
\large{Universidad Sim\'on Bol\'ivar}\\
\large{Departamento de Computaci\'on y Tecnolog\'ia de la Informaci\'on}\\
\large{Laboratorio de Lenguajes de Programaci\'on - CI3661}\\[7cm]
\Huge\textbf{Tarea 2}\\[7cm]
\begin{raggedleft}
\large{Alejandra Cordero - 12-10645}\\
\large{Pablo Maldonado -12-10561}\\
\large Abril - Julio 2016
\end{raggedleft} 

\end{center}

\end{titlepage}

\pagenumbering{arabic}

\paragraph{}

1. Demuestre los teoremas TD y DM por Fitch (teorema de la disyunci\'on y teorema de De Morgan respectivamente)


\begin{itemize}

\item $p \rightarrow q	
\leftrightarrow \neg p \lor q$

\[
\begin{nd}
\hypo {1} {p \rightarrow q	
\leftrightarrow \neg p \lor q}
\open
\hypo {2} {p \rightarrow q}
\open
\hypo {3} { \neg ( \neg p \lor q )}
\open
\hypo {4} {p}
\have {5} {q} \ie{1,3}
\have {6} {\neg p \lor q} \oi{5}
\have {7} {(\neg p \lor q) \land \neg  (\neg p \lor q ), \bot}   \ai{2,5}
\close
\have {8} { \neg p } 
\have {9} {\neg p \lor q} \oi{8}
\have {10} {(\neg p \lor q) \land \neg  (\neg p \lor q ),  \bot}   \ai{5,8}
\close
\have {11} {\neg p \lor q}
\close
\open
\hypo {12} {\neg p \lor q}
\open
\hypo {13} {p}
\have {14} {\neg p \lor q} \r{11}
\open
\hypo {15} {\neg p} 
\open 
\hypo {16} {\neg q}
\have {17} {p} \r{12}
\have {18} {\neg p} \r{14}
\have {19} {p \land \neg p,  \bot} \ai{12,14}
\close
\have {20} {q}
\close
\open
\hypo {21} {q}
\have {22} {q}
\close
\have {23} {q}
\close
\have {24} {p \rightarrow q}
\close
\have {25} {p \rightarrow q \leftrightarrow \neg p \lor q}
\close
\end{nd}
\]



\item $\neg (p \land q) \leftrightarrow \neg p \lor \neg q$

Primero se demostrar\'a $\neg (p \land q) \rightarrow \neg p \lor \neg q$

\[
\begin{nd}
\hypo {1} {\neg (p \land q)} 
\open
\hypo {2} {\neg (\neg p \lor \neg q)}
\open
\hypo {3} {p}
\open
\hypo {4} {q}
\have {5} {p \land q} \ai{3,4}
\have {6} {p \land q \land \neg (p \land q),  \bot } \ai{5,1}
\close
\have {7} {\neg q} \ni{6}
\have {8}  {\neg q \lor \neg p} \oi{7}
\have {9}  {\neg q \lor \neg p \land \neg (p \land q), \bot } \ai{9,1}
\close
\have {10} {\neg p} \ni{9}
\have {11} {\neg p \lor \neg q} \oi{10}
\have {12} {\neg p \lor \neg q \land \neg (\neg p \lor \neg q), \bot } \ai{11,2}
\close
\have {13} {\neg p \lor \neg q} \ni{13}
\close
\end{nd}
\]

Ahora se demostrar\'a $\neg p \lor \neg q \rightarrow \neg (p \land q)$
\[
\begin{nd}
\hypo {1} {\neg p \lor \neg q} 
\open
\hypo {2} {\neg p}
\open
\hypo {3} {p \land q }
\have {4} {p} \ae{3}
\have {5} {\neg p} \r{2}
\have {6} {p \land \neg p,  \bot} \ai{4,5}
\close
\have {7} {\neg (p \land q) } \ni{6}
\close
\open
\hypo {8} {\neg q}
\open
\hypo {9} {p \land q }
\have {10} {q} \ae{9}
\have {11} {q \land \neg q,  \bot} \ai{9,10}
\close
\have {12} {\neg (p \land q) } \ni{11}
\close
\have {13} {\neg (p \land q) } \oe{1,7,12}
\end{nd}
\]
		
Como se demostr\'o que se cumple: \
\item $\neg (p \land q) \rightarrow \neg p \lor \neg q$ y \
\item $\neg p \lor \neg q \rightarrow \neg (p \land q)$ \
entonces, en efecto queda demostrado
$\neg (p \land q) \leftrightarrow \neg p \lor \neg q$
\end{itemize}

\paragraph{}

2. Las reglas en PROLOG son f\'ormulas en FNC que adem\'as deben ser cl\'ausulas de Horn ?`Qu\'e significa este concepto? Adem\'as deben estar "skolemizadas"?`Por qu\'e? ?`Qu\'e hace ese procedimiento? Por \'ultimo existe la unificaci\'on ?`En qu\'e consiste?

\paragraph{Cl\'ausula de Horn}\mbox{}\\

En primer lugar, se desea saber qu\'e es una Clausula de Horn. Para ello, hace falta definir primero los siguientes t\'erminos:

\begin{itemize}

\item Literal: Es una f\'ormula at\'omica o su negaci\'on. Por ejemplo, son literales: $p, \neg q, z, \neg t$, etc.
\item Cl\'ausula: Disyunci\'on de literales. Ejemplo: $ p \lor \neg q \lor \neg t \lor ... \lor z $ 

\end{itemize}

Luego, una Clausula de Horn no es m\'as que una cl\'ausula, en la cual existe exactamente un literal positivo. Por ejemplo: $ \neg p \lor \neg q \lor ... \lor \neg t \lor u$ 

\paragraph{Skolemizaci\'on}\mbox{}\\





\paragraph{Unificaci\'on}\mbox{}\\

Procedimiento mediante el cual Prolog "matches" dos t\'erminos.


\paragraph{}

3. Demuestre por Fitch y Resoluci\'on por refutaci\'on

\begin{itemize}

	\item $\{A \rightarrow (( B \lor \neg A) \land D), E \rightarrow (\neg B \land \neg C), (B \land G) \rightarrow (\neg C \land \neg D)\} \vdash G  \rightarrow ((A \lor C) \rightarrow ( \neg B \land \neg E))$

	\begin{itemize}
	
	\item M\'etodo de Fitch:
	
	
\[
\begin{nd}
\have {1} {A \rightarrow (( B \lor \neg A) \land D)}
\have {2} {E \rightarrow (\neg B \land \neg C)}
\have {3} {(B \land G) \rightarrow (\neg C \land \neg D)}
\hypo {4} {G}
\open
\hypo {5} {A \lor C}
\open
\hypo {6} {A}
\open
\hypo {7} {B \lor E}
\have {8} {( B \lor \neg A) \land D} \ie{1,6}
\have {9} {D} \ae{8}
\have {10} { B \lor \neg A} \ae{8}
\have {11} {A \rightarrow B,   Teorema TD} 
\have {12} {B} \ie{11,6}
\have {13} {B \land G} \ai{12,4}
\have {14} {\neg C \land \neg D} \ie{3,13}
\have {15} {\neg D} \ae{14}
\have {16} {D \land \neg D,  \bot} \ai{9,15}
\close
\have {17} {\neg (B \lor E)}
\have {18} {\neg B \land \neg E, Teorema DM}

\close
\open
\hypo {19} {C}
\have {20} {E \rightarrow (\neg B \land \neg C)} \r{2}
\have {21} {\neg E \lor (\neg B \land \neg C), Teorema TD}
\have {22} {\neg E \lor \neg (B \lor C), Teorema TD}
\have {23} {(B \lor C) \rightarrow \neg E, Teorema TD}
\have {24} { B \lor C } \oi{19}
\have {25} {\neg E} \ie{23,24}
\have {26} {(B \land G) \rightarrow (\neg C \land \neg D)} \r{3}
\have {27} {\neg (B \land G) \lor (\neg C \land \neg D), Teorema TD} 
\have {28} {\neg (B \land G) \lor \neg (C \lor D), Teorema DM }
\have {29} {(C \lor D) \rightarrow \neg (B \land G), Teorema DM }  
\have {30} { C \lor D } \oi{19}
\have {31} {\neg (B \land G)} \ie{29,30} 
\have {32} {\neg B \lor \neg G, Teorema DM } 
\have {33} {G \rightarrow \neg B, Teorema TD } 
\have {34} {\neg B} \ie {33,4}
\have {35} {\neg B \land \neg E} \ai{34,25}
\close
\have {35} {\neg B \land \neg E} \oe{5,18,35}

\close
\end{nd}
\]
	
	
	
	
	
	
	
	
	
	
	
	
	\item M\'etodo de Resoluci\'on por Refutaci\'on:
	
		\paragraph{}
		En primer lugar, es necesario transformar la expresi\'on a la FNC correspondiente 	
	
	
	
		\begin{prooftree}
		\AxiomC{$p \lor q $}
		\AxiomC{$p \lor \neg q $}
		\AxiomC{C}
		\BinaryInfC{D}
		\BinaryInfC{E}
		\end{prooftree}
	
	\end{itemize}		
	
	\item  $\neg p \rightarrow (q \lor r)) \vdash ((\neg p \rightarrow \neg q) \lor r)$ 
	
	\begin{itemize}
	
	\item M\'etodo de Fitch:
	
	\item M\'etodo de Resoluci\'on por Refutaci\'on:
		\paragraph{}
		En primer lugar, es necesario transformar la expresi\'on a la FNC correspondiente
	
		\begin{prooftree}
		\AxiomC{$p \lor q $}
		\AxiomC{$p \lor \neg q $}
		\AxiomC{C}
		\BinaryInfC{D}
		\BinaryInfC{E}
		\end{prooftree}
	
	\end{itemize}	
	
	
	
\end{itemize}


\paragraph{Referencias} 

\begin{itemize}

\item 

\end{itemize}

\end{document}


